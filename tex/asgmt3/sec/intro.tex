\section{Introduction}
Nowadays, image classification has become a very hot topic. 
It refers to the task of categorizing a digital image into given classes. 
To do this, a general workflow is needed.
First, a dataset that contains the image data is needed, in which images should be labeled by their classes. 
The data set will then be divided into two major parts, the training set, and the test set.
The features of images from both training set and test set should be extracted before the classification.
This process is called feature extraction.
Feature extraction helps to reduce the calculations of the classifier and improve the accuracy, 
by means of generating lower dimension descriptions of images.
After feature extraction, the feature can be used as the input to the classifier to train and test the model.
The training set is used to train the classifier in the process showed in fig \ref{fig:proc:train}, 
\begin{figure}[H]
  \centering
  \begin{tikzpicture}[node distance=2cm, scale=0.5]
  \node[process](ts){Training Set};
  \node[process, below left of=ts, yshift=-1cm, xshift=-1cm](fe){Feature Extraction (SIFT)};
  \node[process, below right of=ts, yshift=-3cm](tc){Train Classifier (SVM)};
  \draw[arrow](ts) to node{images} (fe);
  \draw[arrow](fe) to node {features} (tc);
  \draw[arrow](ts) to node {labels} (tc);
  \end{tikzpicture}
  \caption{Training Process}
  \label{fig:proc:train}
\end{figure}
which will then be evaluated by the test set in the process showed in fig \ref{fig:proc:test}.

\begin{figure}[H]
\centering
  \begin{tikzpicture}[node distance=2cm]
  \node[process](ts){Test Set};
  \node[process, below of=ts](fe){Feature Extraction (SIFT)};
  \node[process, below of=fe](tc){Prediction (Trained SVM)};
  \draw[arrow](ts) to node{images} (fe);
  \draw[arrow](fe) to node {features} (tc);
  \end{tikzpicture}
  \caption{Test Process}
  \label{fig:proc:test} %% label for second subfigure
\end{figure}

In this assignment, we use scale-invariant feature transform (SIFT) as the feature extraction method and support vector machine (SVM) as the classifier.


