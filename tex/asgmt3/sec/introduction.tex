\section{Introduction}
Nowadays, image classification has become a very hot topic. 
It refers to the task of categorizing a digital image into given classes. 
To do this, a general workflow should be followed.
First, a dataset that contains the image data is required, in which images should be labeled by their classes. 
Then the data set will be divided into two sets, the training set, and the test set.
Next, the features of images from both training set and test set will be extracted by SIFT.
After this, feature will be clustered using the k-means algorithm.
Steps above help to reduce the amount of calculation of the classifier and improve the accuracy of it by generating lower dimension descriptions of images.
Finally, a vector which describe a pictures will be achieved and it will be input into the classifier (SVM) to train or test the model.
Following graphs show the process of training and test respectively.

\begin{figure}[H]
\begin{minipage}[t]{0.5\linewidth}
  \centering
  \begin{tikzpicture}[node distance=2cm, every node/.style={scale=0.75}]
  \node[io](ts){Input (training set)};
  \node[process, below left of=ts, yshift=-1cm, xshift=-1cm](fe){Feature Extraction (SIFT)};
  \node[process, below of=fe](fc){Feature Clustering(k-means)};
  \node[process, below right of=ts, yshift=-2cm, xshift=2cm](ec){Encode};
  \node[process, below right of=ts, yshift=-6cm](tc){Train Classifier (SVM)};

  \draw[arrow](ts) to node{images} (fe);
  \draw[arrow](fe) to node {features} (fc);
  \draw[arrow](fc) to node {input vectors} (tc);
  \draw[arrow](ts) to node {labels} (ec);
  \draw[arrow](ec) to node {label vectors} (tc);
  \end{tikzpicture}
  \caption{Training Process}
  \label{fig:proc:train}

\end{minipage}%
\begin{minipage}[t]{0.5\linewidth}
  \centering
  \begin{tikzpicture}[node distance=2cm, every node/.style={scale=0.75}]
  \node[io](ts){Input (test set)};
  \node[process, below of=ts](fe){Feature Extraction (SIFT)};
  \node[process, below of=fe](fc){Feature Clustering (k-means)};
  \node[process, below of=fc](tc){Prediction (trained SVM)};
  \node[process, below of=tc](dc){Decode};
  \node[io, below of=dc](res){Output (result)};

  \draw[arrow](ts) to node{images} (fe);
  \draw[arrow](fe) to node {features} (fc);
  \draw[arrow](fc) to node {input vectors} (tc);
  \draw[arrow](tc) to node {label vectors} (dc);
  \draw[arrow](dc) to node {labels} (res);
  \end{tikzpicture}
  \caption{Test Process}
  \label{fig:proc:test} %% label for second subfigure

\end{minipage}
\end{figure}

% \begin{figure}[H]
%   \centering
%   \begin{tikzpicture}[node distance=2cm, every node/.style={scale=0.5}]
%   \node[io](ts){Input (training set)};
%   \node[process, below left of=ts, yshift=-1cm, xshift=-1cm](fe){Feature Extraction (SIFT)};
%   \node[process, below of=fe](fc){Feature Clustering(k-means)};
%   \node[process, below right of=ts, yshift=-2cm, xshift=2cm](ec){Encode};
%   \node[process, below right of=ts, yshift=-6cm](tc){Train Classifier (SVM)};

%   \draw[arrow](ts) to node{images} (fe);
%   \draw[arrow](fe) to node {features} (fc);
%   \draw[arrow](fc) to node {input vectors} (tc);
%   \draw[arrow](ts) to node {labels} (ec);
%   \draw[arrow](ec) to node {label vectors} (tc);
%   \end{tikzpicture}
%   \caption{Training Process}
%   \label{fig:proc:train}
% \end{figure}

% \begin{figure}[H]
% \centering
%   \begin{tikzpicture}[node distance=2cm, every node/.style={scale=0.5}]
%   \node[io](ts){Input (test set)};
%   \node[process, below of=ts](fe){Feature Extraction (SIFT)};
%   \node[process, below of=fe](fc){Feature Clustering (k-means)};
%   \node[process, below of=fc](tc){Prediction (trained SVM)};
%   \node[process, below of=tc](dc){Decode};
%   \node[io, below of=dc](res){Output (result)};

%   \draw[arrow](ts) to node{images} (fe);
%   \draw[arrow](fe) to node {features} (fc);
%   \draw[arrow](fc) to node {input vectors} (tc);
%   \draw[arrow](tc) to node {label vectors} (dc);
%   \draw[arrow](dc) to node {labels} (res);
%   \end{tikzpicture}
%   \caption{Test Process}
%   \label{fig:proc:test} %% label for second subfigure
% \end{figure}




