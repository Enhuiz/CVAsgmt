\section{Classifier: Support Vector Machine}
\subsection{Overview} 
\textbf{Support Vector Machine (SVM)} is a popular classifier proposed first in 1963. 
It was only capable of the linear classification problem until 1992, a kernel trick was invented to deal with the nonlinear classification. 
In the following sections, we will first show how the linear classification works and then dive into the kernel trick.

\subsection{Linear Classification}
Suppose we have the linear separable data (fig \ref{fig:svm:lsd}). 
\begin{figure}[H]
\centering

\begin{tikzpicture}
\centering
\begin{axis}[
    axis lines=middle,
    xmin=0, xmax=6,
    ymin=0, ymax=6,
    xtick=\empty, ytick=\empty
]
\addplot [only marks, mark=-, mark size=3, semithick] table {
1 2
2 1
};
\addplot [only marks, mark=+, mark size=3, semithick] table {
2.5 2.5
3   3
};
\end{axis}
\end{tikzpicture}

\caption{The linear separable data}
\label{fig:svm:lsd}
\end{figure}

We want to choose a separation so that the margin between two groups can be maximized (fig \ref{fig::svm:mg}). 

\begin{figure}[H]
\centering

\begin{tikzpicture}
\centering
\begin{axis}[
    axis lines=middle,
    xmin=0, xmax=6,
    ymin=0, ymax=6,
    xtick=\empty, ytick=\empty
]
\addplot [only marks, mark=-, mark size=3, semithick] table {
1 2
2 1
};
\addplot [only marks, mark=+, mark size=3, semithick] table {
2.5 2.5
3 3
};

\addplot [domain=1.2:3.8, samples=4, color=red, thick] {-x+5};
\draw[<->, thick] (axis cs:1.55, 1.55) -- (axis cs:2.43, 2.43);
\addplot [domain=0.2:2.8, samples=4, color=red, thick] {-x+3};
\end{axis}
\end{tikzpicture}

\caption{Maximized margin}
\label{fig:svm:mg}
\end{figure}

To find the maximized margin, we should write the two line as linear equation. 
We want to define the boundaries of margin (i.e. the red lines) symmetrically. A way to do this is:
\begin{align}
\begin{split}
\mathbf{w}^T \mathbf{x} + b = \gamma \\
\mathbf{w}^T \mathbf{x} + b = -\gamma
\end{split}
\end{align}

The $\gamma$s are so cumbersome. 
As the $w$ and $b$ is arbitrary, we can just substitute $w$ with $\frac{w}{\gamma}$ and $b$ with $\frac{b}{\gamma}$, 
then we get:

\begin{align}
\begin{split}
\mathbf{w}^T \mathbf{x} + b = 1 \\
\mathbf{w}^T \mathbf{x} + b = -1
\end{split}
\end{align}

Now we have two boundaries. We can easily calculate the distance between them:

\begin{equation}
\mathrm{distance} = \frac{2}{||\mathbf{w}||}
\end{equation}

Our target is to maximized this distance in the premise that no sample is located in the margin, which means:

\begin{align}
\begin{split}
& \max_{\mathbf{w},\mathbf{b}} \ \frac{2}{||\mathbf{w}||} \\ 
s.t. \quad & \mathbf{w}^T \mathbf{x}_+ + b \geq 1, \forall \mathbf{x}_+ \in P\\
 & \mathbf{w}^T \mathbf{x}_- + b \leq -1 , \forall \mathbf{x}_- \in N
\end{split}
\label{eq:svm:t1}
\end{align}
Where $P$ is the set of all positive samples and $N$ is the set of all negative samples.
For mathematical convenience, we define $y$:

\begin{equation}
y_i = 
\begin{cases}  
+1,& \mathbf{x}_i \text{ is } \mathbf{x}_+\\
-1,& \mathbf{x}_i \text{ is } \mathbf{x}_-
\end{cases}
\end{equation}

And instead of trying to maximizing the $\frac{2}{||\mathbf{w}||} $, we minimize $\frac{||\mathbf{w}||^2}{2}$. 
The transforms above are carefully designed to make the later calculations easier. 
Now we can rewrite equation \ref{eq:svm:t1} in a simpler way:

\begin{align}
\begin{split}
& \min_{\mathbf{w},\mathbf{b}} \ \frac{||\mathbf{w}||^2}{2} \\ 
s.t. \quad & y_i(\mathbf{w}^T \mathbf{x}_i + b) \geq 1, \forall \mathbf{x}_i \in P \cup N \\ 
\end{split}
\end{align}

This is a convex optimal problem with inequality constraints, which can be solved by solving its Lagrange dual problem 
(Sadly, the Lagrange dual problem was not taught in the mathematical course of SSE, SCUT).
The dual problem can be represented as below:

\begin{align}
\begin{split}
L_D(\mathbf{\lambda}) = \inf_{\mathbf{w}, b} \ L_P(\mathbf{w}, b, \mathbf{\lambda}) =& \inf_{\mathbf{w}, b} \left( \frac{1}{2}||\mathbf{w}||^2 - \sum_i\lambda_i y_i(\mathbf{w}^T\mathbf{x}_i+b) + \sum_i\lambda_i \right)\\
&\max_\mathbf{\lambda} \ L_D(\mathbf{\lambda}) \\
\label{eq:svm:dl1}
\end{split}
\end{align}

Then we fix the $\mathbf{w}$ and $b$ to calculate the infimum. 
To do this, the partial derivative should be calculated and set to $0$:

\begin{align}
\begin{split}
\frac{\partial{L_P}}{\partial{\mathbf{w}}} =& \mathbf{w} - \sum_i\lambda_iy_i\mathbf{x}_i = 0 \\
\frac{\partial{L_P}}{\partial{b}} =& -\sum_i\lambda_iy_i = 0
\end{split}
\end{align}

Substitute $\mathbf{w}$ in equation \ref{eq:svm:dl1}, we get:

\begin{align}
\begin{split}
&\max_\mathbf{\lambda} \ L_D(\mathbf{\lambda}) = \sum_i\lambda_i - \frac{1}{2} \sum_i\sum_j\lambda_i\lambda_jy_iy_j\mathbf{x}_i^T\mathbf{x}_j\\
 &s.t. \ \sum_i\lambda_iy_i = 0, \lambda_i \geq 0, \forall i
\label{eq:svm:dl2}
\end{split}
\end{align}

$\mathbf{\lambda}$ can be calculated based on the algorithm called \textbf{sequential minimal optimization algorithm}. After training, the model can be used to do linear binary classification:

\begin{align}
\begin{split}
\mathbf{w}^T \mathbf{u} + b =  \left(\sum_i\lambda_iy_i\mathbf{x}_i^T\mathbf{u}\right) + b 
\end{split}
\end{align}

\section{Non-linear Classification}
For data inseparable, a \textbf{kernel function} to map the data to a higher dimension solves this problem. 
To do this, we rewrite the term $ \mathbf{x}_i^T\mathbf{x}_j $in equation \ref{eq:svm:dl2} to $K(\mathbf{x}_i^T, \mathbf{x}_j )$:

\begin{align}
\begin{split}
&\max_\mathbf{\lambda} \ L_D(\mathbf{\lambda}) = \sum_i\lambda_i - \frac{1}{2} \sum_i\sum_j\lambda_i\lambda_jy_iy_j K(\mathbf{x}_i^T, \mathbf{x}_j )\\
 &s.t. \ \sum_i\lambda_iy_i = 0, \lambda_i \geq 0, \forall i
\label{eq:svm:dl3}
\end{split}
\end{align}

where K is the kernel function, which first casts the samples, i.e. $\mathbf{x}_i $ and $ \mathbf{x}_j$, into higher dimension then calculate their dot product.
There are several common kernel functions such as polynomial kernel function: 
$$K(\mathbf{x}, \mathbf{y}) = (\mathbf{x}^T\mathbf{y} + 1) ^ p$$
radial basis function:
$$K(\mathbf{x}, \mathbf{y}) = e ^\frac{- ||\mathbf{x} -\mathbf{y}||^2}{2\sigma^2}$$
sigmoid function:
$$K(\mathbf{x}, \mathbf{y}) = \tanh(\kappa \mathbf{x}^T\mathbf{y} - \delta)$$

So far we get the kernel function, it helps us deal with the non-linear separable binary classification.
Image classification problem is generally non-linear separable problem, a kernel must be used when classify them.