%%%%%%%%%%%%%%%%%%%%%%%%%%%%%%%%%%%%%%%%%
% Short Sectioned Assignment
% LaTeX Template
% Version 1.0 (5/5/12)
%
% This template has been downloaded from:
% http://www.LaTeXTemplates.com
%
% Original author:
% Frits Wenneker (http://www.howtotex.com)
%
% License:
% CC BY-NC-SA 3.0 (http://creativecommons.org/licenses/by-nc-sa/3.0/)
%
%%%%%%%%%%%%%%%%%%%%%%%%%%%%%%%%%%%%%%%%%

%----------------------------------------------------------------------------------------
%	PACKAGES AND OTHER DOCUMENT CONFIGURATIONS
%----------------------------------------------------------------------------------------

\documentclass[paper=a4, fontsize=11pt]{scrartcl} % A4 paper and 11pt font size

\usepackage[T1]{fontenc} % Use 8-bit encoding that has 256 glyphs
\usepackage{fourier} % Use the Adobe Utopia font for the document - comment this line to return to the LaTeX default
\usepackage[english]{babel} % English language/hyphenation
\usepackage{amsmath,amsfonts,amsthm} % Math packages

\usepackage{lipsum} % Used for inserting dummy 'Lorem ipsum' text into the template

\usepackage{sectsty} % Allows customizing section commands
\allsectionsfont{\centering \normalfont\scshape} % Make all sections centered, the default font and small caps

\usepackage{fancyhdr} % Custom headers and footers
\pagestyle{fancyplain} % Makes all pages in the document conform to the custom headers and footers
\fancyhead{} % No page header - if you want one, create it in the same way as the footers below
\fancyfoot[L]{} % Empty left footer
\fancyfoot[C]{} % Empty center footer
\fancyfoot[R]{\thepage} % Page numbering for right footer
\renewcommand{\headrulewidth}{0pt} % Remove header underlines
\renewcommand{\footrulewidth}{0pt} % Remove footer underlines
\setlength{\headheight}{13.6pt} % Customize the height of the header

\numberwithin{equation}{section} % Number equations within sections (i.e. 1.1, 1.2, 2.1, 2.2 instead of 1, 2, 3, 4)
\numberwithin{figure}{section} % Number figures within sections (i.e. 1.1, 1.2, 2.1, 2.2 instead of 1, 2, 3, 4)
\numberwithin{table}{section} % Number tables within sections (i.e. 1.1, 1.2, 2.1, 2.2 instead of 1, 2, 3, 4)

\setlength\parindent{0pt} % Removes all indentation from paragraphs - comment this line for an assignment with lots of text

%----------------------------------------------------------------------------------------
%	TITLE SECTION
%----------------------------------------------------------------------------------------

\newcommand{\horrule}[1]{\rule{\linewidth}{#1}} % Create horizontal rule command with 1 argument of height

\title{	
\normalfont \normalsize 
\textsc{South China University of Technology, Department of Software Engineering} \\ [25pt] % Your university, school and/or department name(s)
\horrule{0.5pt} \\[0.4cm] % Thin top horizontal rule
\huge Computer Vision Assignment 1 \\ % The assignment title
\horrule{2pt} \\[0.5cm] % Thick bottom horizontal rule
}


\author{Niu Zhe \\201430613093} % Your name

\date{\normalsize\today} % Today's date or a custom date

\begin{document}

\maketitle % Print the title

%----------------------------------------------------------------------------------------
%	PROBLEM 1
%----------------------------------------------------------------------------------------

\section{feature extraction algorithm}

\subsection{Scale-invariant feature transform}

\paragraph{}
Scale-invariant feature transform (SIFT) is an algorithm to detects local features in images. 
It is proposed by David Lowe in 1999, and was patented by University of British Columbia.

\paragraph{}
SIFT has several advantages: 
\begin{enumerate}
\item Locality: features are local, making it robust to occlusion and clutter.
\item Quantity: even a small objects can generate many features
\item Efficiency: nearly real-time performance
\end{enumerate}

\paragraph{}
The algorithm can be divided into four stages: 
Scale-space extreme detection, keypoint localization, orientation assignment and keypoint descriptor. 

\subsubsection{Scale-space extreme detection}

The image is first down-sampled and convolved with Gaussian filter.
$$
L(x, y, \sigma) = G(x, y, \sigma) * I(x, y)
$$
where 
$L$ is the blurred image, 
$G$ is the Gaussian Blur operator, 
$I$ is the original image, 
$x, y$ is the pixel location, 
$\sigma$ is the scale parameter, greater the value, greater the blur.
By using performing Gaussian filter with different scale parameter several times on different down-sampled images, new imaged are generated.
The images are then divided into different octaves according to the original down-sampled image.

\paragraph{}
Next, the difference of images in the same octave will be taken, this is called Difference of Gaussian (DoG). 
Keypoints are then taken as the maxima or minima points of DoG $D(x, y, \sigma)$ that occur at different octave.

$$
D(x, y, \sigma) = L(x, y, k_i\sigma) - L(x, y, k_{i+1}\sigma)
$$
where $k_i$, $k_{i+1}$ is the scale parameter of convolved image $i$.

\subsection{Keypoint Localization}
The extreme points are usually too many, some of which are unstable.
Keypoint localization need to be performed to reject points that have low contrast or are poorly localized along an edge.

$$
D(\mathbf{x}) = D + \frac{\partial D }{\partial x}\mathbf{x} + \frac{1}{2}\mathbf{x^T}\frac{\partial ^ 2 D}{\partial \mathbf{x}^2}\mathbf{x}
\mathbf{x} = (x, y, \sigma)^T
$$
minima or maxima is located at 
$$
\mathbf{x}\hat = -\frac{\partial ^ 2 D^{-1}}{\partial \mathbf{x}^2}\frac{\partial D }{\partial x}
$$
value of $D(\mathbf{x})$ must be large, $|D(\mathbf{x})| > threshold$.

\begin{align} 
\begin{split}
(x+y)^3 	&= (x+y)^2(x+y)\\
&=(x^2+2xy+y^2)(x+y)\\
&=(x^3+2x^2y+xy^2) + (x^2y+2xy^2+y^3)\\
&=x^3+3x^2y+3xy^2+y^3
\end{split}					
\end{align}

Phasellus viverra nulla ut metus varius laoreet. Quisque rutrum. Aenean imperdiet. Etiam ultricies nisi vel augue. Curabitur ullamcorper ultricies

%------------------------------------------------

\subsection{Heading on level 2 (subsection)}

Lorem ipsum dolor sit amet, consectetuer adipiscing elit. 
\begin{align}
A = 
\begin{bmatrix}
A_{11} & A_{21} \\
A_{21} & A_{22}
\end{bmatrix}
\end{align}
Aenean commodo ligula eget dolor. Aenean massa. Cum sociis natoque penatibus et magnis dis parturient montes, nascetur ridiculus mus. Donec quam felis, ultricies nec, pellentesque eu, pretium quis, sem.

%------------------------------------------------

\subsubsection{Heading on level 3 (subsubsection)}

\lipsum[3] % Dummy text

\paragraph{Heading on level 4 (paragraph)}

\lipsum[6] % Dummy text

%----------------------------------------------------------------------------------------
%	PROBLEM 2
%----------------------------------------------------------------------------------------

\section{Lists}

%------------------------------------------------

\subsection{Example of list (3*itemize)}
\begin{itemize}
	\item First item in a list 
		\begin{itemize}
		\item First item in a list 
			\begin{itemize}
			\item First item in a list 
			\item Second item in a list 
			\end{itemize}
		\item Second item in a list 
		\end{itemize}
	\item Second item in a list 
\end{itemize}

%------------------------------------------------

\subsection{Example of list (enumerate)}
\begin{enumerate}
\item First item in a list 
\item Second item in a list 
\item Third item in a list
\end{enumerate}

%----------------------------------------------------------------------------------------

\end{document}